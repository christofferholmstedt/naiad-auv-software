\section{Terms and Definitions}
Terms and definitions are copied from documents related
to these topics. These are the SPA Standard Drafts
\cite{standard:spa_logical_interface,standard:spa_networking,
standard:spa_spacewire_adaptation,standard:spa_physical_interface,
standard:spa_local_adaptation} and the CAN Bus 2.0 Specification
\cite{standard:can_bus}.

\subsection{General}
\begin{itemize}
    \item \textbf{UUID:} Universally Unique Identifier
    \item \textbf{URN:} Uniform Resource Name
    \item \textbf{ASIM:} Appliqu\'{e} Sensor Interface Module
\end{itemize}

\subsection{Space Plug-and-Play Avionics}
\begin{itemize}
    \item \textbf{SM-x:} SPA Subnet Manager, where x represents a given
        technology protocol.
    \item \textbf{CUUID:} Component Universally Unique Identifier.
    \item \textbf{xTEDS:} Extensible Transducer Electronic Data Sheet.
    \item \textbf{XUUID:} xTEDS Universally Unique Identifier.
    \item \textbf{CAS:} The Central Addressing Service (CAS) is
responsible for providing logical address blocks to be assigned to each
hardware or software component. The CAS stores the logical address block
and logical address for each SPA Manager in the SPA Network.
    \item \textbf{Plug-and-Play (PnP):} Ability to connect a device to the larger system and
have the two work together with little or no set-up required (e.g., automated
system recognition and data exchange).
    \item \textbf{SPA Lookup Service:} The SPA Lookup Service is responsible for accepting
component registration and providing data source route information for
components requesting a particular type of service (or returning a nack if
the service is not available).
    \item \textbf{SPA Manager:} The SPA manager is responsible for performing discovery for a
particular subnet. It maps incoming packets to the correct SPA endpoint on the
subnet, encapsulating the SPA packet with the correct protocol header. In the
reverse direction it removes the protocol header and possibly adds a new header
conforming to the subnet the packet is about to enter. It is also responsible
for topology discovery and reporting within the subnet.
    \item \textbf{SPA-L / SPA-Local:} The SPA Local Subnet Adaptation specifies the means
by which SPA components interoperate on a local processing node. It specifies
the inter-process communication method used in order to establish a SPA local
network and the messages, protocols, and interactions required in order to
facilitate PnP functionality.
\end{itemize}

\subsection{CAN Bus}
\begin{itemize}
    \item \textbf{CAN FD:} CAN with flexible data-rate. Supported from Linux kernel 3.6, can
be tested through virtual CAN interfaces.
    %\item \textbf{:}
\end{itemize}
