\section{Introduction}
\textit{The following introduction is the same as the one for SPA Local Subnet Standard
\cite{standard:spa_local_adaptation} with a slight modification in the last
paragraph. SPA Local has been changed to SPA CAN Bus.}
% Perhaps even better to copy information from chapter 1 "Scope" and 4
% "Overview" which gives a better introduction and the introduction.

SPA embraces and implements a collection of standards designed to facilitate
rapid constitution of spacecraft systems using modular components.

The data portion of the standard is based on a data-centric model in which
components self-describe their interfaces when they register with the system.
This self-description is facilitated by an XML byte string called an
extensible Transducer Electronic Data Sheet (xTEDS). The xTEDS defines the
identity of the component, and the organization of its data interfaces.

SPA is a networked data exchange model. One of the premises of SPA is that
there is no distinction between a hardware device that supports a data
interface and a software application that does the same - thus all endpoints
in a SPA network, physical or virtual, are referred to as "components".

Other components may use queries to express their data consumption needs.
These queries are based on the specification of data "kind" with "qualifiers"
that provide additional context to focus the search. When matches are found,
the consumer may subscribe to messages of providers that meet their search
criteria. As such, SPA systems dynamically bind their data at runtime.

The SPA CAN Bus Subnet Adaptation specifies the means by which SPA components
interoperate on a CAN Bus Subnet. It specifies the inter SPA-C component
communication method used in order to establish a SPA CAN Bus Subnet and the
messages, protocols, and interactions required in order to facilitate PnP
functionality.
