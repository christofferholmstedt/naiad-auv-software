\section{The SPA CAN Bus Subnet}\label{ch:spa_can_bus_subnet}
The CAN Bus is specified in different ISO Standards and Bosch has shared their
initial specification available for free on internet \cite{standard:can_bus}.
The main obstacle to solve to be able to set up a SPA-C Subnet is the difference
between SPA and CAN Bus when it comes to addressing components.

In SPA each component must have an address this is the opposite of how a
CAN Bus works. Within a CAN Bus all componenents reads the same message
at the same time, depending on the message id it's dropped or read by specific
components. For the continuation of this CAN Bus adaptation the assumption is
made that it's a viable solution to implement a kind of address resolution
protocol on top of the CAN Bus without loosing too much bandwidth/efficiency.
The rationale behind this is presented in appendix \ref{appendix:bootstraping}.

A SPA CAN Bus Subnet Manager (SM-c) will provide required component discovery
and messaging communication functions.

\subsection{SPA-C Message Composition}
% One bit "fragmentet" set to 1 if it is a fragmentet packet.
text


% Reason somewhere that offset field and more fragments field is not necassary
% due to the CAN Bus physics and 16 bit or at least 13 bit offset field would
% be way too much overhead.

% If no specific SPA-C header is needed then send all data in several messages
% and always finish with an empty message. ~ Nils

% All messages can be fragmented but if a message ends on even 8 bytes
% another empty message (Data Length) must be zero to finish the message.
\subsubsection{SPA Message}
text
\subsubsection{SPA-C Message}
text
\subsubsection{SPA-C Message Header}
text
