% THIS IS SIGPROC-SP.TEX - VERSION 3.1
% WORKS WITH V3.2SP OF ACM_PROC_ARTICLE-SP.CLS
% APRIL 2009
%
% It is an example file showing how to use the 'acm_proc_article-sp.cls' V3.2SP
% LaTeX2e document class file for Conference Proceedings submissions.
% ----------------------------------------------------------------------------------------------------------------
% This .tex file (and associated .cls V3.2SP) *DOES NOT* produce:
%       1) The Permission Statement
%       2) The Conference (location) Info information
%       3) The Copyright Line with ACM data
%       4) Page numbering
% ---------------------------------------------------------------------------------------------------------------
% It is an example which *does* use the .bib file (from which the .bbl file
% is produced).
% REMEMBER HOWEVER: After having produced the .bbl file,
% and prior to final submission,
% you need to 'insert'  your .bbl file into your source .tex file so as to provide
% ONE 'self-contained' source file.
%
% Questions regarding SIGS should be sent to
% Adrienne Griscti ---> griscti@acm.org
%
% Questions/suggestions regarding the guidelines, .tex and .cls files, etc. to
% Gerald Murray ---> murray@hq.acm.org
%
% For tracking purposes - this is V3.1SP - APRIL 2009

\documentclass{acm_proc_article-sp}

% For proper urls in the references list
\usepackage[hyphens]{url}

\begin{document}

\title{Space Plug-and-Play Avionics over CAN Bus\titlenote{This report was
        written in October 2013 in the advanced level project course
        DVA425 at M\"{a}lardalen University, Sweden.}}
%\subtitle{[Extended Abstract]
%\titlenote{A full version of this paper is available as
%\textit{Author's Guide to Preparing ACM SIG Proceedings Using
%\LaTeX$2_\epsilon$\ and BibTeX} at
%\texttt{www.acm.org/eaddress.htm}}}
%
% You need the command \numberofauthors to handle the 'placement
% and alignment' of the authors beneath the title.
%
% For aesthetic reasons, we recommend 'three authors at a time'
% i.e. three 'name/affiliation blocks' be placed beneath the title.
%
% NOTE: You are NOT restricted in how many 'rows' of
% "name/affiliations" may appear. We just ask that you restrict
% the number of 'columns' to three.
%
% Because of the available 'opening page real-estate'
% we ask you to refrain from putting more than six authors
% (two rows with three columns) beneath the article title.
% More than six makes the first-page appear very cluttered indeed.
%
% Use the \alignauthor commands to handle the names
% and affiliations for an 'aesthetic maximum' of six authors.
% Add names, affiliations, addresses for
% the seventh etc. author(s) as the argument for the
% \additionalauthors command.
% These 'additional authors' will be output/set for you
% without further effort on your part as the last section in
% the body of your article BEFORE References or any Appendices.

\numberofauthors{1}
% I've updated the number of authers ~ Christoffer 2012-10-14

%  in this sample file, there are a *total*
% of EIGHT authors. SIX appear on the 'first-page' (for formatting
% reasons) and the remaining two appear in the \additionalauthors section.
%
\author{
% You can go ahead and credit any number of authors here,
% e.g. one 'row of three' or two rows (consisting of one row of three
% and a second row of one, two or three).
%
% The command \alignauthor (no curly braces needed) should
% precede each author name, affiliation/snail-mail address and
% e-mail address. Additionally, tag each line of
% affiliation/address with \affaddr, and tag the
% e-mail address with \email.
%
% 1st. author
\alignauthor
Christoffer Holmstedt\\
       \email{christoffer.holmstedt@gmail.com}
% 2nd. author
% \alignauthor
% firstname surname\\
%        \email{email}
}
% There's nothing stopping you putting the seventh, eighth, etc.
% author on the opening page (as the 'third row') but we ask,
% for aesthetic reasons that you place these 'additional authors'
% in the \additional authors block, viz.
% \additionalauthors{Additional authors: John Smith (The Th{\o}rv{\"a}ld Group,
%email: {\texttt{jsmith@affiliation.org}}) and Julius P.~Kumquat
%(The Kumquat Consortium, email: {\texttt{jpkumquat@consortium.net}}).}
%\date{30 July 1999}
% Just remember to make sure that the TOTAL number of authors
% is the number that will appear on the first page PLUS the
% number that will appear in the \additionalauthors section.

\maketitle
\begin{abstract}
    Abstract. What is the state of the art? Maximum 250 words.
\end{abstract}

% A category with the (minimum) three required fields
% \category{A.1}{General Literature}{Introductory and Survery}
%A category including the fourth, optional field follows...
% If we want to add another category (or several).
%\category{D.2.8}{CHANGE THIS Software Engineering}{Metrics}[complexity measures, performance measures]

% \terms{Theory}

% \keywords{Threat Models, Centralised systems, Decentralised systems, Censorship, Privacy, Natural disasters} % NOT required for Proceedings

\section{Introduction}
\textit{The following introduction is the same as the one for SPA Local Subnet Standard
\cite{standard:spa_local_adaptation} with a slight modification in the last
paragraph. SPA Local has been changed to SPA CAN Bus.}
% Perhaps even better to copy information from chapter 1 "Scope" and 4
% "Overview" which gives a better introduction and the introduction.

SPA embraces and implements a collection of standards designed to facilitate
rapid constitution of spacecraft systems using modular components.

The data portion of the standard is based on a data-centric model in which
components self-describe their interfaces when they register with the system.
This self-description is facilitated by an XML byte string called an
extensible Transducer Electronic Data Sheet (xTEDS). The xTEDS defines the
identity of the component, and the organization of its data interfaces.

SPA is a networked data exchange model. One of the premises of SPA is that
there is no distinction between a hardware device that supports a data
interface and a software application that does the same - thus all endpoints
in a SPA network, physical or virtual, are referred to as "components".

Other components may use queries to express their data consumption needs.
These queries are based on the specification of data "kind" with "qualifiers"
that provide additional context to focus the search. When matches are found,
the consumer may subscribe to messages of providers that meet their search
criteria. As such, SPA systems dynamically bind their data at runtime.

The SPA CAN Bus Subnet Adaptation specifies the means by which SPA components
interoperate on a CAN Bus Subnet. It specifies the inter SPA-C component
communication method used in order to establish a SPA CAN Bus Subnet and the
messages, protocols, and interactions required in order to facilitate PnP
functionality.

\section{Terms and Definitions}
Terms and definitions are copied from documents related
to these topics. These are the SPA Standard Drafts
\cite{standard:spa_logical_interface,standard:spa_networking,
standard:spa_spacewire_adaptation,standard:spa_physical_interface,
standard:spa_local_adaptation} and the CAN Bus 2.0 Specification
\cite{standard:can_bus}.

\subsection{General}
\begin{itemize}
    \item \textbf{UUID:} Universally Unique Identifier
    \item \textbf{URN:} Uniform Resource Name
    \item \textbf{ASIM:} Appliqu\'{e} Sensor Interface Module
\end{itemize}

\subsection{Space Plug-and-Play Avionics}
\begin{itemize}
    \item \textbf{SM-x:} SPA Subnet Manager, where x represents a given
        technology protocol.
    \item \textbf{CUUID:} Component Universally Unique Identifier.
    \item \textbf{xTEDS:} Extensible Transducer Electronic Data Sheet.
    \item \textbf{XUUID:} xTEDS Universally Unique Identifier.
    \item \textbf{CAS:} The Central Addressing Service (CAS) is
responsible for providing logical address blocks to be assigned to each
hardware or software component. The CAS stores the logical address block
and logical address for each SPA Manager in the SPA Network.
    \item \textbf{Plug-and-Play (PnP):} Ability to connect a device to the larger system and
have the two work together with little or no set-up required (e.g., automated
system recognition and data exchange).
    \item \textbf{SPA Lookup Service:} The SPA Lookup Service is responsible for accepting
component registration and providing data source route information for
components requesting a particular type of service (or returning a nack if
the service is not available).
    \item \textbf{SPA Manager:} The SPA manager is responsible for performing discovery for a
particular subnet. It maps incoming packets to the correct SPA endpoint on the
subnet, encapsulating the SPA packet with the correct protocol header. In the
reverse direction it removes the protocol header and possibly adds a new header
conforming to the subnet the packet is about to enter. It is also responsible
for topology discovery and reporting within the subnet.
    \item \textbf{SPA-L / SPA-Local:} The SPA Local Subnet Adaptation specifies the means
by which SPA components interoperate on a local processing node. It specifies
the inter-process communication method used in order to establish a SPA local
network and the messages, protocols, and interactions required in order to
facilitate PnP functionality.
\end{itemize}

\subsection{CAN Bus}
\begin{itemize}
    \item \textbf{CAN FD:} CAN with flexible data-rate. Supported from Linux kernel 3.6, can
be tested through virtual CAN interfaces.
    %\item \textbf{:}
\end{itemize}

\section{Problem Statement}\label{ch:problem_statement}
\subsection{Bootstraping}
When starting a new system the first thing that occurs is the "Network Topology
Discovery" \cite{standard:spa_networking}. This process assigns a logical
address to each component in the network. This contradicts the CAN Bus
specification where no component is identified by a specific address.

The assumption is made that it's more important to implement SPA over CAN Bus
than it is to follow the intentions of the CAN Bus specification. The first
thing that needs to be defined is how should different components be addressed
in a local CAN Bus subnet?


\subsubsection{Random Message ID}
Temp text
\subsection{Number of Message IDs used}
\subsection{Fragmentation}
Traversing the SPA CAN Network.

\section{SPA CAN Bus Adaptation}
Suggested solution...

\section{Conclusions}
\subsection{Future work}
Bootstraping. Instead of assuming that you have to step away from the CAN Bus
specification to make SPA work...that is each component must have its own
address...maybe it's possible for a SM-c to handle many-to-many relations
between SPA components.

All components have a unique SPA logical address but within each CAN Bus subnet
messages can be read by multiple components (that is listening to the same can
bus msg id).


%\end{document}  % This is where a 'short' article might terminate

% Just comment this out if we don't need it.
%\input{acknowledgment.tex}

%
% The following two commands are all you need in the
% initial runs of your .tex file to
% produce the bibliography for the citations in your paper.
\bibliographystyle{abbrv}
\bibliography{sigproc}  % sigproc.bib is the name of the Bibliography in this case
% You must have a proper ".bib" file
%  and remember to run:
% latex bibtex latex latex
% to resolve all references
%
% ACM needs 'a single self-contained file'!
%
%APPENDICES are optional
%\balancecolumns
%\appendix
%Appendix A
%\section{Headings in Appendices}
%The rules about hierarchical headings discussed above for
%the body of the article are different in the appendices.
%In the \textbf{appendix} environment, the command
%\textbf{section} is used to
%indicate the start of each Appendix, with alphabetic order
%designation (i.e. the first is A, the second B, etc.) and
%%a title (if you include one).  So, if you need
%hierarchical structure
%\textit{within} an Appendix, start with \textbf{subsection} as the
%highest level. Here is an outline of the body of this
%document in Appendix-appropriate form:
%\subsection{Introduction}
%\subsection{The Body of the Paper}
%\subsubsection{Type Changes and  Special Characters}
%\subsubsection{Math Equations}
%\paragraph{Inline (In-text) Equations}
%\paragraph{Display Equations}
%\subsubsection{Citations}
%\subsubsection{Tables}
%\subsubsection{Figures}
%\subsubsection{Theorem-like Constructs}
%\subsubsection*{A Caveat for the \TeX\ Expert}
%\subsection{Conclusions}
%\subsection{Acknowledgments}
%\subsection{Additional Authors}
%This section is inserted by \LaTeX; you do not insert it.
%You just add the names and information in the
%\texttt{{\char'134}additionalauthors} command at the start
%of the document.
%\subsection{References}
%Generated by bibtex from your ~.bib file.  Run latex,
%then bibtex, then latex twice (to resolve references)
%to create the ~.bbl file.  Insert that ~.bbl file into
%the .tex source file and comment out
%the command \texttt{{\char'134}thebibliography}.
% This next section command marks the start of
% Appendix B, and does not continue the present hierarchy
%\section{More Help for the Hardy}
%The acm\_proc\_article-sp document class file itself is chock-full of succinct
%and helpful comments.  If you consider yourself a moderately
%experienced to expert user of \LaTeX, you may find reading
%it useful but please remember not to change it.
\balancecolumns
% That's all folks!
\end{document}
