\section{Introduction}
\subsection{Rationale}
\subsubsection{Space Plug-and-Play Avionics (SPA)}
The main goal with Space Plug-and-Play Avionics (SPA) is to make the
assembly process of a spacecraft go as fast as possible. Development of
separate components can take years but when it's time to connect the component
to a system, it should be done in a matter of hours or days.

For this SPA is being defined in a set of standard documents ranging from how
the communication protocol should work to how different types of subnets comply
with the general logical interface standard \cite{standard:spa_logical_interface,
    standard:spa_networking,standard:spa_spacewire_adaptation,standard:spa_physical_interface,
standard:spa_local_adaptation}.

Some key concepts with the Space Plug-and-Play technology
\begin{itemize}
    \item Assembly of components to a complete system should be done in hours or days.
    \item SPA defines a set of technologies that can be used for communication;
        this includes, but is not limited to SPA-u for "USB subnets" and SPA-s
        for SpaceWire subnets.
\end{itemize}

% As Lyke et. al. says
% \textit{"One must not confuse the idea of building systems fast with the idea of
% creating systems that can be built fast"} \cite{lyke2010}.
% that makes it possible


\subsubsection{CAN Bus}
The CAN Bus was created by Bosch for the automotive industry \cite{standard:can_bus}, it's designed to
reduce wiring and have a high-level of safety built-in. There is no routing
system in a CAN Bus so all connected nodes can read all messages and send as
soon as the bus is free. Instead all types of messages have a Message ID linked
to it and all nodes that want to read that kind of message filter them out.

The Message IDs also defines the priority in a CAN Bus. If two nodes starts to
send a message at the same time, arbitration will be done by comparing all bits
in the message id. As soon as a node detects that another message is being sent
at the same time with a lower Message ID it will stop transmitting.

Some key concepts with the CAN Bus
\begin{itemize}
    \item All nodes can read all messages.
    \item Multiple nodes are not allowed to send messages with the same message
        ID.
    \item Message IDs are used for priority.
\end{itemize}




\subsection{SPA over CAN Bus}
This report will go through the problems with adapting SPA for the CAN Bus as
well as possible solutions. This work is done as a part of a project at
M\"{a}lardalens University where an Autonomous Underwater Vehicle is built.
The main motiviation for "SPA over CAN Bus" are the requirements from the
customer for this project. Throughout the report basic knowledge about the CAN
Bus as well as the general concepts of SPA is assumed.

The report is structured as follows. It starts with terms and definitions,
followed by problem statement, solutions are then given in the "SPA CAN Bus Adaptation"
chapter. The conclusions from this research is given in the end as well as
if suggested solutions are worth going forward with in the Autonomous
Underwater Vehicle project.
